\documentclass[10pt]{article}
% \usepackage{geometry}
% \geometry{margin=0.2in}
% \usepackage[X2]{fontenc}
\usepackage[utf8]{inputenc}
% \usepackage[utf8x]{inputenc}

\nonstopmode
% \usepackage{minted}[cache=false]
\usepackage{graphicx} % Required for including pictures
\usepackage[figurename=Figure]{caption}
\usepackage{float}    % For tables and other floats
\usepackage{amsmath}  % For math
\usepackage{amssymb}  % For more math
\usepackage{fullpage} % Set margins and place page numbers at bottom center
\usepackage{paralist} % paragraph spacing
\usepackage{subfig}   % For subfigures
%\usepackage{physics}  % for simplified dv, and 
\usepackage{enumitem} % useful for itemization
\usepackage{siunitx}  % standardization of si units
\usepackage{hyperref}
\usepackage{mmacells}
\usepackage{listings}
\usepackage{svg}
\usepackage{xcolor, soul}
\usepackage{bm}
% \usepackage{amsthm}  % For math
\usepackage{mathtools}

% \usepackage{setspace}
% \usepackage{listings}
% \usepackage{listings}
% \usepackage[autoload=true]{jlcode}
% \usepackage{pygmentize}



\usepackage[margin=1.8cm]{geometry}
\newcommand{\C}{\mathbb C}
\newcommand{\D}{\bm D}
\newcommand{\R}{\mathbb R}
\newcommand{\Q}{\mathbb Q}
\newcommand{\Z}{\mathbb Z}
\newcommand{\N}{\mathbb N}
\newcommand{\PP}{\mathbb P}
\newcommand{\A}{\mathbb A}
\newcommand{\F}{\mathbb F}
\newcommand{\1}{\mathbf 1}
\newcommand{\ip}[1]{\left< #1 \right>}
\newcommand{\abs}[1]{\left| #1 \right|}
\newcommand{\norm}[1]{\left\| #1 \right\|}

\def\Tr{{\rm Tr}}
\def\tr{{\rm tr}}
\def\Var{{\rm Var}}
\def\calA{{\mathcal A}}
\def\calB{{\mathcal B}}
\def\calD{{\mathcal D}}
\def\calE{{\mathcal E}}
\def\calG{{\mathcal G}}
\def\from{{:}}
\def\lspan{{\rm span}}
\def\lrank{{\rm rank}}
\def\bd{{\rm bd}}
\def\acc{{\rm acc}}
\def\cl{{\rm cl}}
\def\sint{{\rm int}}
\def\ext{{\rm ext}}
\def\lnullity{{\rm nullity}}
\DeclareSIUnit\clight{\text{\ensuremath{c}}}
\DeclareSIUnit\fm{\femto\m}
\DeclareSIUnit\hplanck{\text{\ensuremath{h}}}
\usepackage[cache=false]{minted}

% \usepackage{ tipa }

\DeclareUnicodeCharacter{2208}{\ensuremath{\in}}
\DeclareUnicodeCharacter{2082}{\ensuremath{\phantom{}_2}}
\DeclareUnicodeCharacter{03A3}{\ensuremath{\Sigma}}
\DeclareUnicodeCharacter{03C0}{\ensuremath{\pi}}
\DeclareUnicodeCharacter{03C3}{\ensuremath{\sigma}}
\DeclareUnicodeCharacter{03C4}{\ensuremath{\tau}}
\DeclareUnicodeCharacter{0394}{\ensuremath{\Delta}}



\definecolor{mintedbackground}{rgb}{0.902, 0.929, 0.906}

\definecolor{cambridgeblue}{rgb}{0.81, 0.9, 0.84}



\sethlcolor{mintedbackground}
\newcommand{\mathcolorbox}[1]{\colorbox{mintedbackground}{$\displaystyle #1$}}

% \lstdefinelanguage{julia}%
%   {morekeywords={abstract,break,case,catch,const,continue,do,else,elseif,%
%       end,export,false,for,function,immutable,import,importall,if,in,%
%       macro,module,otherwise,quote,return,switch,true,try,type,typealias,%
%       using,while},%
%    sensitive=true,%
% %    alsoother={$},%
%    morecomment=[l]\#,%
%    morecomment=[n]{\#=}{=\#},%
%    morestring=[s]{"}{"},%
%    morestring=[m]{'}{'},%
% }[keywords,comments,strings]%

% \lstset{%
%     language         = Julia,
%     basicstyle       = \ttfamily,
%     keywordstyle     = \bfseries\color{blue},
%     stringstyle      = \color{magenta},
%     commentstyle     = \color{ForestGreen},
%     showstringspaces = false,
% }

% $
\begin{document}
\begin{center}
	\hrule
	\vspace{.4cm}
	{\textbf { \large CDS DS 210 --- Programming Data Science}}
\end{center}
{\textbf{Name:}\ Emmy Blumenthal \hspace{\fill} Final Project Report\hspace{\fill}  \textbf{BU ID:} \ U87312711 \\
\textbf{Due Date:}\  Dec 15, 2022   \hspace{\fill} \textbf{Email:}\ emmyb320@bu.edu \ 
\vspace{.4cm}
\hrule

\begin{center}
	\Large 
	{
	\bf
	Implementing Karger's Algorithm on the Facebook Social Circles Graph Dataset
	}
\end{center}
\section*{Karger's Algorithm}
% \section*{REPORT}

% % \url{https://snap.stanford.edu/biodata/datasets/10023/10023-CC-Neuron.html}

% The structure and differentiation of different cell types is a central question in biology and biophysics that has a history of combining epistemological questions (e.g., what is a cell type?) and quantitative methods.
% In this project, I will employ graph clustering methods to try to approach some of these questions from a basic level.
% I will investigate the `megascale cell-cell similarity network' from the Stanford network analysis project (\url{https://snap.stanford.edu/biodata/datasets/10023/10023-CC-Neuron.html}).
% Specifically, I will attempt to use a variety of graph clustering algorithms to find subgraphs/cliques of the similarity network which represent cells that are more similar to each other than other clusters of cells; an example algorithm I will implement is the highly connected subgraphs (HCS) clustering algorithm (\url{https://en.wikipedia.org/wiki/HCS_clustering_algorithm}).
% The relative success of the algorithm at finding independent clusters may tell me something about cell types.
% If clusters are sufficiently independent and there are many clusters, this tells us that cell types may be well-differentiated.
% On the other hand, if there are few clusters and/or it is difficult to find independent cliques/clusters, I might be able to conclude that cell types are difficult to differentiate using this method.

% If I am able to find sufficiently differentiated clusters, a next possible step would be implementing vertex centrality measures which may tell us which cells can serve as archetypes/representatives for their given cell type.
% These centrality measures would be implemented within the individual cell sub-graphs.
% Because individual cell's gene expressions often fluctuate significantly, it is likely that finding archetypal example cells may be difficult/infeasible.
% However, understanding the way in which various centrality methods may succeed or fail to identify only a few archetypes for each subgraph may provide some insights into how cell types could be defined and their structures of differentiation.
% Possible centrality measures that I may employ include closeness and eigenvector centrality.

% To accomplish this project, I will attempt to adhere to the following schedule:
% \begin{itemize}
% 	\item Week of Nov. 18: Download data and ensure that I can read data in the necessary formats (e.g., adjacency matrix, sparse representation)
% 	\item Week of Nov. 25: Give a first pass at implementing the HCS clustering algorithm; identify questions and errors that will need to be tackled after Thanksgiving break.
% 	\item Week of Dec. 2: Polish HCS clustering algorithm and seek support for persistent issues.
% 	Assess feasibility of implementing sub-graph centrality measures.
% 	\item Week of Dec. 9:
% 	If implementing sub-graph centrality measures is feasible given implemented HCS algorithm, implement two centrality measures and assess results relative to each other.
% 	If implementing sub-graph centrality measures is unsuccessful, attempt to implement a different sub-graph/clique clustering algorithm besides HCS.
% 	\item Week of Dec. 15:
% 	Finalize report on results; clean repository and source code.
% \end{itemize} 










\end{document}





